\section{Further Work}
The following are outlines for solutions to some of
the problems identified in our implementation.

\subsection{External Core Parser}
As mentioned in the evaluation section we provide no means of
parsing GHC's external core into the representation our compiler accepts.
Implementing this parser is not especially difficult. The Haskell
algebraic data type we use as our input language representation is
almost identical to the official definition GHC provide
for their own algebraic data type representing Core. Examples of
parsers written for previous versions of Core can be found in old
versions of GHC. As Core and its tools have not been well maintained 
in recent versions, these parsers do not quite fit the corre language
specification. However it should be entirely possible to adapt the 
last such parser to accept current core and convert it to into our 
algebraic data type representation. Even if this is not possible, Core
is designed to be parseable and is built from a very small set of 
components. Writing a parser from scratch would be a little 
time-consuming, but not especially difficult.

\subsection{Characters, String}
Adding characters and strings would not be particularly difficult.
We could add a node type to our Gmachine representing characters if
we wished. Alternatively, we could represent characters as their
ascii integer representations with the existing NNum node type. 
Strings could be built in much the same way we currently handle
lists; A string could be constructed as a list of character nodes
(or NNum node character represenations) concatenated together. 

\subsection{Browser Interaction Methods}
In order to allow for the writing of useful browser executed programs
our language would need a set of primitive operations corresponding
to common browser actions. Such actions would include DOM manipulation
methods, input handling etc. A number of such examples exist in the
DSL of Fay examined earlier. There are two obvious ways we could add such 
functionality to our compiler. The first would be to add instructions to
handle primitive operations. Much like the existing instructions for
arithmetic, equality etc. these would be emitted when we encounter 
certain keywords in our input language. A list of supercombinators
representing the browser actions we wish to represent would be added
as compiled primitives to our compiler. When our runtime encounters
these instructions it would convert them into the appropriate browser
actions to execute. 

Alternatively, we could write a DSL in the Haskell we compile to
represent these actions. In much the same way as Fay, we'd use Haskell
type and data type definitions to correspond to the browser actions
we wish to represent. Our compiler and runtime would maintain a list
of these definitions and the browser actions they correspond to. When
the runtime encounters such a definition, it would execute the corresponding
browser method.

\subsection{Better Compilation Scheme}
We investigated two compilation schemes before implementing our compiler.
The latter of these, the Gmachine, was chosen for its various advantages
as outlined in the design choices section. However, there exist other
schemes which may suit the purposes of this project better. Of note is
the Three Instruction Machine, a compilation scheme which uses only
three instructions in its runtime and opts for a \emph{spineless} graph 
representation in the heap. Such an implemenation would be of obvious
interest to us, as it would allow for a very minimal runtime. This would
improve load times for pages using our compiled language as there would
be less data to transmit when loading the runtime. However, this comes
at a tradeoff. The initial state is more complicated and evaluation will
require more steps than our Gmachine implemenation. Such considerations
would ultimately decide the choice of best compilation scheme for our
purposes. 

\subsection{Garbage Collection}
A significant issue in our implementation is the lack of a garbage collection
system for our heap. This would prove to be quite problematic (even fatal)
for any large program we wish to run. A lot of research has gone into
functional language garbage collection and i would found it an intersting
feature to implement. Sadly there was not enough time and it was decided
that a compiler without a garbage collector would be of more use than a 
garbage collector without a compiler. A garbage collection implementation
exists for our Gmachine implementation in the Jones and Lester implemenation
tutorial. Further work related to garbage collection would probably start
there, unless an alternative compilation scheme was used.
