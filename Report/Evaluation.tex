\section{Evaluation}
We have examined the design and implementation of our solution.
We will now evaluate its usefulness and the degree to which it
solves the problems we identified earlier. We have already seen
an example compilation and evaluation of a simple program containing
our "K" function. This program demonstrated a number of the solutions
we wished to implement.

\subsection{Completeness}
Firstly, we were capable of taking a representation of a functional
language derived from Haskell and compiling it into an initial state.
This representation is capable of expressing
a number of features, of which we support the following:

\begin{itemize}
	\item Primitive integer values
	\item Conditionals
	\item Data constructors
	\item Case statements
	\item Expressions built from the above
\end{itemize}

\noindent The input language also supports let expressions, lambda
abstractions, characters and strings however our compiler is not
currently able to handle these. It would not be particularly 
difficult to add support for these however there simply wasn't
enough time. Information on possible solutions will be provided
in the further work section. We may not have a Core parser but
the representation we do accept is general enough to represent
the majority of basic functional language concepts. We originally
intended for this project to use Haskell as its base but it would
not be a massive step to adapt our project to languages such as
Miranda, ML and Lisp. Such languages can easily be traced back to
polymorphic lambda calculus dialects~\cite[pp.10]{SystemF} and as such 
can all be expressed, albeit in somewhat minimal forms, in 
representations very similar to that of our input language.

The initial state we compile this representation into is that of
the Gmachine compilation scheme. This gives us a heap representing
our program from which our evaluator can build a reducible expression
graph. It also provides a stack and state dump which will be 
required by the runtime evaluator. Our compiler can then serialize
this state into an equivalent JavaScript representation which we can
evaluate in a browser, given the right set of abstract definitions
and methods.

These definitions and methods exist in our runtime, also written
in JavaScript. This runtime takes our serialized initial state
and, as demonstrated, evaluates it through a number of subsequent
states until a final state is reached. This final state represents
our fully evaluated program and from this we can extract the result
of evaluation. The features described above can now be written
in a functional language representation and, after compilation
and serialization, be executed in a browser. 

In order for our project to be a viable option when writing 
real-world programs, we would need to add a number of basic browser
interaction components. We can currently evaluate expressions but
not effect the state of the browser. Possible solutions to this
will be outlined in the further work section.

\subsection{A Core Parser}
There is currently no functionality in my project to parse GHC's
external core into the representation provided in GADT.hs.As 
deadlines approached, features yet to be implemented had to be
prioritized. Writing a parser for Core was deemed to be of less
importance and interest than other features. Core is designed to
be easily parsed. Its specification has changed subtly but frequently
over the last decade through various updates to GHC. Completed
parsers exist for previous versions of Core however sometime between
version 5 and the current version of GHC, maintenance of the core
language became less of a priority for the development team. As
a result, tools such as the parser stagnated. Writing a core parser
would not be a difficult task and will be explored in the further
work section below. However, it was a task deemed uninteresting 
when there existed other outstanding work to be completed which was
more directly related to the goal of this project. 


\subsection{Efficiency}
Efficiency was not the main aim of this project. As such, there
exists a lot of scope for optimization and improvement. The simple
examples provided will execute in acceptable time in a browser. 
However, the execution time of more complex programs will certainly 
become noticeably and problematically slower than programs with 
equivalent semantics written in optimized JavaScript. It would 
not be possible to achieve what we wanted and compete directly with
the efficiency of native JavaScript code. This is an inherent flaw
of trying to represent alternative semantics in a target language.
This aside, there is still much unexplored potential in make our
solution more efficient. The compilation scheme we chose is not
in-efficient, but more modern schemes have been developed since
which are more efficient. The JavaScript code used in the runtime
could also stand to be more efficient. When implementing the 
runtime i was most concerned with producing a clean, understandable
proof-of-concept. There is currently no garbage collection in my
implementation. Possible solutions to these will be examined in
the further work section.

\subsection{Some Examples}
We will now provide some examples of some slightly more complicated
programs than the "K" program we used earlier. Firstly, a
factorial function, as is traditional in functional programming. The
following is a factorial function in Haskell:

\begin{verbatim}
    module MIdent where

    fac 0 = 1
    fac n = n * fac (n - 1)

    main = fac 3
\end{verbatim}

This compiles into the following unedited GHC external Core:

\begin{verbatim}
%module main:MIdent
  %rec
  {main:MIdent.fac :: ghczmprim:GHCziTypes.Int ->
                      ghczmprim:GHCziTypes.Int =
     \ (dsdl0::ghczmprim:GHCziTypes.Int) ->
         %case ghczmprim:GHCziTypes.Int dsdl0
         %of (wildX4::ghczmprim:GHCziTypes.Int)
           {ghczmprim:GHCziTypes.Izh (ds1dl1::ghczmprim:GHCziPrim.Intzh) ->
              %case ghczmprim:GHCziTypes.Int ds1dl1
              %of (ds2Xl6::ghczmprim:GHCziPrim.Intzh)
                {%_ ->
                   base:GHCziNum.zt @ ghczmprim:GHCziTypes.Int 
                   base:GHCziNum.zdfNumInt
                   wildX4
                   (base:GHCziNum.zm @ ghczmprim:GHCziTypes.Int
                    base:GHCziNum.zdfNumInt (main:MIdent.fac wildX4)
                    (ghczmprim:GHCziTypes.Izh (1::ghczmprim:GHCziPrim.Intzh)));
                 (0::ghczmprim:GHCziPrim.Intzh) ->
                   ghczmprim:GHCziTypes.Izh (1::ghczmprim:GHCziPrim.Intzh)}}};
  main:MIdent.main :: ghczmprim:GHCziTypes.Int =
    main:MIdent.fac
    (ghczmprim:GHCziTypes.Izh (4::ghczmprim:GHCziPrim.Intzh));
\end{verbatim}

\noindent When we shorten the type annotations of the above for clarity,
this looks like:

\begin{verbatim}
%module main:MIdent
  %rec
  {fac :: Int -> Int =
     \ (d0::Int) -> %case Int d0 %of (wildX4::Int)

           {Types.Izh (d1::Int) -> %case Types.Int d1 %of (d6::Int)

                 {%_ ->
                    * @ Int NumInt wildX4 (- @ Types.Int Int (fac wildX4) 
                        (Types.Izh (1::Int));
                 
                 (0::Int) ->
                    Types.Izh (1::Int)}}};

main :: Types.Int =
    fac
    (Types.Izh (4::Int));
\end{verbatim}

\noindent When we pass the Haskell ADT representation of this function
through our compilation and serialization phases, we get the following 
JavaScript representation of our compiled initial state:

\begin{verbatim}

var start = new Date().getTime();

var GmOutput = [];

var GmCode = [new PushGlobal("main"),new Eval()];

var GmStack = [];

var GmDump = [];

var GmHeap = {
objCount:12,
freeAddrs:[13..],
addrObjMap:{
12:new NGlobal(3,[new Push(0),new Eval(),
			      new Cond([new Push(1)],
			     [new Push(2)]),new Update(3),
			      new Pop(3),new Unwind()]),
11:new NGlobal(2,[new Push(1),new Eval(),new Push(1),
				  new Eval(),new Neq(),new Update(2),
				  new Pop(2),new Unwind()]),
10:new NGlobal(2,[new Push(1),new Eval(),new Push(1),
				  new Eval(),new Eq(),new Update(2),
				  new Pop(2),new Unwind()]),
9:new NGlobal(1,[new Push(0),new Eval(),new Neg(),new Update(1),
				 new Pop(1),new Unwind()]),
8:new NGlobal(2,[new Push(1),new Eval(),new Push(1),new Eval(),
	  			 new Div(),new Update(2),new Pop(2),new Unwind()]),
7:new NGlobal(2,[new Push(1),new Eval(),new Push(1),new Eval(),
				 new Mul(),new Update(2),new Pop(2),new Unwind()]),
6:new NGlobal(2,[new Push(1),new Eval(),new Push(1),new Eval(),
				 new Sub(),new Update(2),new Pop(2),new Unwind()]),
5:new NGlobal(2,[new Push(1),new Eval(),new Push(1),new Eval(),
			     new Add(),new Update(2),new Pop(2),new Unwind()]),
4:new NGlobal(1,[new Push(0),new Eval(),new Update(1),new Pop(1),new Unwind()]),
3:new NGlobal(0,[new PushInt(3),new PushGlobal("fac"),new Mkap(),
				 new Eval(),new Update(0),new Pop(0),new Unwind()]),
2:new NGlobal(1,[new Push(0),new PushGlobal("xeq0"),new Mkap(),new Eval(),
				 new Cond([new PushInt(1)],[new PushInt(1),new Push(1),
				 new PushGlobal("-"),new Mkap(),new Mkap(),new PushGlobal("fac"),
				 new Mkap(),new Eval(),new Push(1),new Eval(),new Mul()]),
				 new Update(1),new Pop(1),new Unwind()]),
1:new NGlobal(1,[new Push(0),new Eval(),new PushInt(0),new Eq(),new Update(1),
				 new Pop(1),new Unwind()])}
};

var GmGlobals = {"xeq0":1,"fac":2,"main":3,"Id":4,"+":5,
				 "-":6,"*":7,"/":8,"neg":9,"==":10,"!=":11,
				 "if":12};

var GmState = [GmOutput, GmCode, GmStack, GmDump, GmHeap, GmGlobals]; 	


function main(){
return evalProg(GmState);
}

\end{verbatim}

\noindent This can be executed in a browser with a call to \verb!main()!, assuming
the runtime is included. Finally, this will evaluate to a final state in which
the only value left on the stack is a pointer to an NNum 6 node. \verb!main()! will
return the value 6 as removed from that node.