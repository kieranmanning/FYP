
\section{Existing Solutions}
We are not the first to identify these problems, nor to attempt
to rectify them. Much research and work has gone directly into studying
issues in JavaScript. Many of the problems of JavaScript also exist
in the general programming domain where solutions and alternatives
likewise exist. Having looked at some of the problems in JavaScript 
and its place in browser side web programming, we will now examine
some of these existing solutions.


\subsection{Fay}
The Fay language, which lives at https://github.com/faylang/fay/wiki, is
another language which has attempted to solve some of the problems we
are interested in. It provides a Haskell DSL with primitives corresponding
to those of JavaScript as well as various built-in methods to aid in browser
side programming. Programs are written in this DSL then compiled using the
Fay compiler into a javascript representation thereof. The choice of Haskell
as the platform language gives Fay a number of useful properties. Programs 
compiled with Fay are:

\begin{itemize}
	\item Type-safe
	\item Pure
	\item Lazy
\end{itemize}

\noindent The choice of JavaScript as the target language means that programs
compiled with Fay are also compatible with all major web browsers. In order
to evaluate compiled Fay programs, a number of primitive operations
are needed. These are provided in a seperate runtime written in JavaScript.

\subsubsection{A Haskell DSL}
Fay source programs are written in Haskell, making use of a number of
primitives defined in a haskell embedded domain specific language.
In general, there are a number of advantages to using domain specific languages
to accomplish tasks. Such languages can take advantage of the tools and 
features of their platform language. This reduces the time taken to
design and implement the language itself, as the great bulk of the necessary
compilation tools probably already exist for the platform language~\cite{DSLs}. 
The domain specific language (DSL) will typically use constructs in the
platform language to represent the data which we wish to work it. Methods
will be provided to execute common actions on these representations allowing
us to process the data our DSL represents. 

The Fay domain specific language provides a number of important 
constructs. Most significantly, it provides access to the Fay () monad
which represents Fay computations. Functions which are intended to interact
with the browser make use of this monad to execute Fay operations. For
example, the hello world function in Fay might look like this:

\begin{verbatim}
main :: Fay ()
main = alert "Hello, World!"

alert :: String -> Fay ()
alert = ffi "window.alert(%1)"
\end{verbatim}

The type signature of the function \emph{alert} shows this in action.
We say that the alert function takes a String and returns a Fay () action.
In this case, the Fay action is a foreign function interface call to
JavaScripts window.alert() method.  Fay also contains a list of type 
definitions corresponding to the JavaScript types into which our Haskell
and Fay types will be compiled. In this way, primitive Haskell types and 
Fay types can be mixed in Fay source programs. 

This approach allows the programmer to take advantage of many of Haskell's
existing features when writing Fay programs. The syntax remains the same,
with the addition of a few keywords and constructs added by Fay. Someone
coming from a Haskell background with seperate web developmente experience
would have little difficult getting up to speed with Fay, once they figure
out the Fay specific additions in the DSL. As Fay is embedded within Haskell,
it is able to take full advantage of its type system. This means that Fay
programs are type-safe. It also means that error messages and warnings from 
Haskell compilers can be used when debugging Fay programs. This is a significant
improvement over writing JavaScript where debugging errors in such a weakly
typed language tend to be tedious and vague. 

\subsubsection{Runtime}
Fay programs are compiled into a low level JavaScript representation. The
compiled program is built from primitive thunk objects and operations
which process these thunks. The abstract representation for these thunks 
exists in a seperate runtime written in JavaScript. This runtime also contains
various operations to handle thunks in compiled programs along with thunk-level 
functions in JavaScript which represent Fay's primitive operations. When a
Fay program is compiled the output is bundled with the runtime which is 
capable of evaluating the compiled program. 

The thunk objects used to represent Fay programs are not dissimilar from the
nodes found in the graphs described in the background information section.
Singular expressions are built of thunks and complex expressions are built
from series of thunk, much like our reducible graphs. Thunks are by default
unevalated, or unreduced, and must be \emph{forced} in order to reduce to
weak head normal form. This again should sound familiar after examining the
idea of reducible graph expressions. A thunk is represented in the abstract
by a function as follows:

\begin{verbatim}
// Thunk object.
function $(value){
  this.forced = false;
  this.value = value;
}
\end{verbatim}

Instances of thunks are represented as instantiated function objects.
We also need functions to handle primitive thunk operations. Two important
examples are: 

\begin{verbatim}
// Force a thunk (if it is a thunk) until WHNF.
function _(thunkish,nocache){
  while (thunkish instanceof $) {
    thunkish = thunkish.force(nocache);
  }
  return thunkish;
}

// Apply a function to arguments (see method2 in Fay.hs).
function __(){
  var f = arguments[0];
  for (var i = 1, len = arguments.length; i < len; i++) {
    f = (f instanceof $? _(f) : f)(arguments[i]);
  }
  return f;
}
\end{verbatim}

The first forces a thunk to weak head normal form (using a JavaScript
prototype "force" method tied to the thunk object). This is used to 
evaluate expressions represented in thunks to weak head normal form,
much in the manner explained in the introduction section on graph
reduction. The second example here applies a function to two arguments.
We can see this function as being similar to the appliation nodes
we'd find in reducible graph expressions. 

Primitive operations in the runtime are also represented as JavaScript
functions. For example, the two functions needed for primitive
multiplication look like this:

\begin{verbatim}
function Fay$$mult$36$uncurried(x,y){
    return new $(function(){
      return _(x) * _(y);
    });
}

function Fay$$mult(x){
  return function(y){
    return new $(function(){
      return _(x) * _(y);
    });
  };
}
\end{verbatim}

The first represents a fully saturated multiplication operation ie. one
in which we have both necessary arguments. We \emph{return a function
which when called} will force both operands into weak head normal form
and multiply them together. As such, the return of the multiplication
operation is not itself in weak normal form and will remain unevaluated
until it is forced. Such behaviour allows for laziness in the language.

The second function represents an undersaturated multiplication operation.
In this case, we have only one of the two necessary arguments. We take
our provided operand and place it into wrapper function which takes a second
operand. This function contains the means to return an unevaluated
saturated multiplication operation. We then return the wrapper function, 
which when applied to the as of yet unprovided second operand, will return
an unevaluated multiplication expression, just as in the full saturated example.
Such behaviour is allows for curried operations in Fay.

There are many other functions in the runtime although these examples
are sufficient to get a feel for how it works. A large part of the runtime
is just implementations of primitives such as arithmetic and list operations
etc. There are also two other somewhat significant functions which handle
the serialization and unserialization of objects from Fay to JS. However
these are mostly utility functions and are not particularly interesting
to us, once we have seen the principles of the implemenation.

\subsubsection{Analysis}
Fay's take on the problem of improving browser side functional programming
presents some interesting ideas. In terms of accomplishments, it manages
to bring the ideas of laziness, static typing and purity to the problem
space while also remaining compatible with all current browsers. Its
implementation also provides some interesting concepts. Of note in particular
is its runtime, which provides a means to evaluate a compiled program in a
browser. This is an idea which will be important when it comes time to write
our own implementation. Its use of a Haskell embedded domain specific 
language, and the benefits this brings, is also interesting to us.

\subsection{iTasks}
iTasks is a project built with the Clean programming language. Its aim
is different from ours but its implementation manages to achieve some of
our goals. iTasks is a workflow management system specification language,
embedded in Clean, which generates workflow applications. As it is written
as a DSL inside a functional language, much like Fay, it benefits from a
number of the advantages of the host language. In particular, iTasks inherits
Clean's concepts of generic programming and its type system.

\subsubsection{Implementation}
The domain specific language of iTasks makes available to the programmer a
number of constructs and functions needed to express a series of workflow
tasks when specifying a workflow system. Tasks are represented by a 
parameterized \verb!Task a!
type and functions can be constructed in the domain specific language which
take these tasks and perform operations on them. A set of basic functions
provides functionality such as task composition, splitting etc. The following
is an example of a basic iTasks specification

\begin{verbatim}
:: Person = { firstName   :: String, 
              SurName     :: String,
              dateOfBirth :: HtmlDate,
              gender      :: Gender }
:: Gender = Male | Female

enterPerson :: Task Person
enterPerson = enterInformation "Enter Information"
\end{verbatim}

From this, iTasks will produce a standard form with fields for firstName, 
surName, dateOfBirth and gender. There are a number of interesting points
in this example. Clean's concept of records is used to define a "Person",
consisting of a number of fields of types String, HtmlDate and Gender. 
Gender is also a user defined type. The function enterPerson takes no
parameters and returns a value of type Task Person, which is a Task
carrying data of type Person. This is making use of Clean's concept 
of parameterized data types to carry aditional information about the 
task in particular. As well as iTasks specific functions, the programmer
can write general Clean code to assist in their work with iTasks.

\subsection{Core/SAPL paper} 


\subsection{Others}
We have discussed some existing projects which provide solutions to 
some of our problems at length. It is worth mentioning a few other 
projects in passing which are related to our work. As previously stated,
this is far from the first project to identify problems with the state
of browser-side programming and reliance on JavaScript. Many projects
have been created to address these problems and many take similar 
approaches to the ones discussed above; using JavaScript as a target
language, adding to its semantics and/or avoiding certain unwanted 
aspects.

Coffeescript is an example of such a project which has gained a lot
of interest in the last few years. First released in late 2009, 
CoffeeScript provides a sort of wrapper language which compiles into
equivalent JavaScript. The underlying semantics of JavaScript are
unchanged. The aim of the project was to bring a more pythonic 
syntax to the language. It achieves this, making it possible to 
write what would normally be verbose code in JavaScript as 
smaller, equivalent code in CoffeeScript. Declaring a function for
example, which would look like this in JavaScript:

\begin{verbatim}
race = function() {
  var runners, winner;

  winner = arguments[0], 
           runners = 2 <= arguments.length ? 
          __slice.call(arguments, 1) : [];
  return print(winner, runners);
};
\end{verbatim}

\noindent Could be expressed as this in CoffeeScript: 

\begin{verbatim}
race = (winner, runners...) ->
  print winner, runners
\end{verbatim}

\noindent Coffeescript is not of great interest to us however as it has no
effect on the underlying semantics of JavaScript much less bringing
functional concepts to the language.

Another language worth mentioning is ClojureScript. It provides a compiler
which compiles Clojure into JavaScript. This allows the use of Clojure concepts
such as hashmaps, let bindings etc. in browser executed programs by 
representing them in compiled JavaScript. Clojure's syntax is also preserved,
still showing the nested parenthesized expression syntax inherited from and
associated with Lisp. However, my hope was to achieve similar goals with 
Haskell and as literature on ClojureScript is somewhat sparse, it was not
of great use.