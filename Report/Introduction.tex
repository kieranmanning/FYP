
\section{Introduction}
It would help to explain some of the concepts to which I will be referring 
throughout this report. Not everyone will be familiar with some of the 
functional programmming, and more specificially Haskell, ideas being discussed.
While many programmers will have an idea of the workings of javascript, at
least in so far as it is an imperative language, some of the concepts which 
we will be examining might go beyond the scope of casual javascript programming.
For brevity's sake, I will explain some of the more specific concepts which 
I will be commonly referring to throughout this report.

\subsection{Types in Javascript}
**Is this really needed?**
\\
There are seven data types in javascript, five of which concern us for the purposes
of this project; three primary data types which are Number, String and Boolean, 
and two 'composite' data types, Objects and Arrays. When writing in straight
forward javascript, types are more or less invisible to the programmer. To declare
a variable, of any type, the usual format is...
\begin{verbatim}
var varname = value;
\end{verbatim}
The type of varname is inferred from the type of the value. The types of javascipt
objects are inferred from their attributes and methods. Types can be mixed
in operators and functions with some type coercion, which can be dangerous
if used incorrectly. The composite Object type will be most interest to us later,
when dealing with the javascript 'runtime' used both in our own implementation
and in that of the Fay language, which we will be looking at.

An important point to remember is that we can represent primitive objects as
composite objects, albeit it with a performance penalty. This idea will prove
significant later when dealing with lazy evaluation, where simple primitive
types will be insufficient to represent lazy evaluation.

\subsection{Functional Languages}


\subsection{Functional Compilation}
I will be referring frequently to some of the more common ideas and practices
invovled in compiling a functional language thoughout this report. Unsurprisingly,
the field of functional compiler design is not easily summarized in a paragraph,
let alone a paper; the implementation explained here only scratches the surface.
However, it will be easier to understand some of the points I make if we start
with a quick overview.

Most compiler implementations for simple functional languages start with an 
input language which consists of a list of function definitions and some entry
or "main" function, which serves as the entry point for the runtime or evaluator.
We wish to avoid writing programs with side effects and as such the syntax of
our input program will reflect that. There is no concept of global state nor
stateful computations and when we need to represent some concept of changing
state we do so by passing state as paramaters between functions. 

The input language is parsed into a simplified expression-based representation.
One such representation, the one used later in this report, is defined as:

\begin{verbatim}
Program ::= [EVar, [a], Expr a]

Expr a
  ::= EVar Name
    | ENum Int
    | EConstr Int Int
    | EAp (Expr a) (Expr a)
    | ELet 
        IsRec
        [(a, Expr a)]
        (Expr a)
    | ECase
        (Expr a)
        [Alter a]
    | ELam [a] (Expr a)
\end{verbatim}

\noindent We say that a program in this representation consists of a list of
super combinator definitions. A super combinator, for our purposes, is an 
expression body along with an identifying variable name and a list of arguments
to the expressions. Expressions can consist of

\begin{enumerate}
	\item EVars. Variable names corresponding to a defined value.
	\item ENums. Primitive integers
	\item EConstrs. Data constructors, along with the number of elements in the
		  fully saturated constructor as well as the constructors tag, used to
		  differentiate constructors in a given context.
	\item EAp. Binary function applications which we use to apply exprs in order
		  to construct larger exprs.
	\item ELet. Let expressions, allowins us to define a list of arguments local
		  to a specific expression.
	\item ECase. Case expressions, containing first a condition expression which
		  will evaluate to a tag identifying a data constructor in the context
		  of the case expressions and secondly, a list of constructor tags and 
		  the code to be executed when a particular tag is found.
	\item ELam. Lambdas, or anonymous expressions. Note the similarity to a 
		  super combinator, without the identifying variable name.
\end{enumerate}

\noindent By way of example, a sample program taking two paramaeters and returning
their sum, written in such a language, would resemble

\begin{center} \( ["sum"], ["x", "y"], EAp (EAp (EVar "+") (EVar "x")) (EVar "y") \) \end{center}

\noindent Note in particular that we address a supercombinator named "+" 
representing our integer addition primitive and that we use two EAp 
binary expression applications in order to call it on two parameters x and y.

Our compiler will take a program in such a simplified representation
and convert it into an initial state representing our input program and the
means to evaluate it to a final state. Depending on the compiler implementation
in particular, this may consist just of an initial state or an initial state
and a sequence of instructions denoting actions to be executed on the state. 
An interesting 
distinction between these compilation schemes and those of imperative languages
is that we can view our initial (and continuing) program states as graphs built
from a small set of nodes, namely

\begin{enumerate}
	\item Integer nodes, representing literal integer values
	\item Function nodes consisting of the arity of the function and the
		  its behaviour along with the instructions needed to evaluate it.
	\item Application functions, the only nodes capable of forming 
		  node connections from a graph point of view and used to represent
		  expression applications in terms of our simplified expr language.
	\item Data Constructor nodes, consisting of the arity of the data
		  constructor and a list of addresses to the nodes containing its
		  elements.
\end{enumerate}

We say that our program can be represented by a graph of reducible expressions
or \emph{redexes} built from the above nodes. 
Our program state will consist of a heap containing the nodes which form
or graph addressable by their address in the heap. We will
also have a stack used to contain a working set of heap addresses which 
will be required in the evaluation of immediately pending expressions. A 
'globals' dictionary will associate global function identifiers with the
addresses of their function nodes in the heap. \\

**GIVE EXAMPLE OF GRAPH**

**EXAMPLE OF REDEX REDUCTION** \\

An evaulator or runtime evaluates our compiled program by iterating through
states from the initial state to the final. On each iterartion, the state
(and instruction sequence if present) are examined to determine the actions
to execute in order to reach the next state. The appropriate actions are
executed, producing a next state which we will then evaluate in the same
manner. 

\subsection{Lazy Evaluation}
Lazy evaluation, also known as call-by-need evaluation, is an evaluation method
that delays evaluations until such time as their values are
required. The result of this is that a computation we write will only be 
evaluated when it is required by some other aspect of the program, or possibly never
if its value is never required. 

This is useful both from an efficiency point of view and when trying to 
represent concepts which might not fit into stricter, more finite ideas of
programming. 

For the former, it is easiest to demonstrate with an example...
\begin{verbatim}
take 1 ['a'..'e']
\end{verbatim}
Here, \emph{['a'..'z']} represents a list of characters beginning with 'a' and 
ending with 'e'. The function \emph{take} evaluates and returns the number of
elements of the second parameter specified by the first parameter, starting at the
head of the list. In this example we're 'taking' one element from "abcde" which
of course gives us the character 'a'. But what if we were to say...
\begin{verbatim}
take 1 [1..]
\end{verbatim}
As you may have guessed, \emph{[1..]} is the list of integers beginning with 1
and continuing to infinite. Strange concept? In strict evaluation, attempting
to make use of this would be difficult, to say the least. However, with lazy 
evaluation we only need to evaluate as much of the list as we actually need.
The result being that our take function will only evaluate and return the 
first item, ignoring the remainder of the list. What's more, it can do this
as quickly with an infinite list as with any other length list. This is where the
efficiency bonus comes in! **Is that right?**. \\


**EXPLAIN IMPLEMENTATION RE: OUR GRAPH REDUCTION**  \\


A second characteristic often associated with lazy evaluation is that of 
updating. When an expression is evaluated in a lazy context, and we know
that the contents of the expression will not change, we can overwrite the
expression with its final value. Implementation wise, this generally involves
replacing the pointer we previously associated with the expression in the
heap with a pointer to an indirection node pointing to the result of 
evaluating the expression. From then on, if our program attempts to 
access the expression, it will be redirected to a value in the heap 
representing the evaluated expression. This means we need only evaluate
the expression once. 

This chractecteristic, known as sharing, is closely related to the idea
of \emph{referential transparency}. Referential transparency is the name
given to a property of certain expressions which states that the expression
can be replaced with its value without altering the semantics of the overall
program. This property exists in programs which are \emph{pure}, that is to
say programs consisting of expressions or functions whose evaluation is 
dependant only on the body of the expression and it's parameters, and which
cause no side effects which may effect other expressions upon execution. 
We cannot update an expression with its value if there is a danger that
some aspect of the expression may change during the course of program 
execution. This creates certain challenges when dealing with non-deterministic
concepts such as IO in functional languages. Such problems however are 
beyond the scope of this project and implementation. Those interested can
find more information here **LINK**

\subsection{Weak Head / Normal Form}
\emph{Normal form}, and \emph{weak head normal form} or WHNF, are terms used
in this context to denote the level of possible evaluation in an expression.
We say an expression is in head normal form when it has been fully evaluated
or reduced, and can neither be further evaluated nor contains any sub
expressions which can be further evaluated. For example...
\begin{verbatim}
1
\x -> x
(1, 2)
\end{verbatim}
are an integer, lambda expression and tuple respectively in normal form. "1 + 1"
by comparison would not be in normal form as there is an addition operation to
be performed.

Expressions in weak head normal form are expressions that have been evaluated
to the outermost data constructor. These can be data constructors, undersaturated
primitive operations or lambda abstractions. For example...
\begin{verbatim}
Just (1 + 1)
\x -> 2 + 2
'h' : ("e" ++ "ello")
\end{verbatim}
The first example contains two sub expressions which could be further evaluated,
but as the outermost component (the Just data constructor) has been evaluated 
fully, it is still in WHNF. The same is true of the 3rd example, where the "++"
sub expression could be further evaluated, however the outermost element
':' is a data constructor, specifically the list cons constructor. The second
example is in WHNF by virtue of being a lambda abstraction.


**REDEX EXAMPLE**