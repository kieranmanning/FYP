\section{Introduction}
An important aspect of web development is the ability to write dynamic
websites which will adapt to users requirements, input etc. This can be
achieved by writing adaptive client-side or server-side web programs
which serve dynamic content. Client-side browser executed programming 
has become  increasingly important as it is more scalable and allows 
for truly dynamic content without the need for page refreshing. The 
ubiqitous standardized programming language for client-side is JavaScript.
This is the only language which can be executed on all major browsers and 
that seems unlikely to change in the forseeable future. 

JavaScript was created in 1995 for Netscape by Brendan Eich. It is an
object-orientated, imperative, interpreted language intended originally
to be executed by a web browser (although server-side implementations
also exist, notably Node.js in recent years). Despite the name, JavaScript
is unrelated to Java. They both share a similar \verb!C++!-inspired
curly-brace syntax and object orientation but that is about as deep as
the simalirities go. JavaScript is weakly, dynamically typed. Its objects
are not class based but prototyped based. It is interpreted, meaning
that the human readable source code is executed by the browser without
any pre-processing. 

Some features have been added to JavaScript over the years but in 
general the language has not greatly changed since its inception. This
is understandably problematic. As of the moment of this writing the
language is 18 years old. 
As web development and websites have become increasingly more complex
problems inherent in JavaScript have started to become more obvious.
JavaScript's syntax is unappealing and verbose. Its type system is 
weak in more than name alone, leading to frequent problems with ill-
written JavaScript programs breaking silently. Many such problems 
would have been detected or prevented in languages with stronger
type systems. JavaScript lacks any concept of a module system making
inclusion difficult and naive. Programs are "included" in web pages
by linking to their sources which are basically concatenated by the
browser executing the page. Such issues existed from the first 
release of the language. As the websites and the JavaScript
programs on which they rely have grown larger, unimaginably so since
1995, these issues have become increasingly problematic.

Many people have attempted to
write alternative languages which could replace JavaScript but 
introducing new standards in the hopes of replacing old standards
usually ends as predicted. As JavaScript began as the \emph{the
standard}, so it seems it shall continue. There is some light to the
situation, however. Much time and effort has been put into making 
JavaScript run fast and it is capable of great efficiency. While 
JavaScript may be our only client side programming language, at least
it \emph{does} provide us with a standard upon which all major web 
browser vendors will agree. There have been attempts
to improve JavaScript by augmenting its feature set through the use
of user created libraries. One notable example is JQuery, a JavaScript
library which adds new features and alternative syntax to the language.
These are still only addons though, relying on the underlying semantics
and executing of the original language. 

