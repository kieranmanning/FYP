
\section*{Goal}
The goal of this project is to try and improve on some of the problems currently 
inherent in writing browser executed programs using some of the common features
of functional languages as inspiration. The ubiquitous standard programming 
language for writign browser side web programs is javascript. Javascript can produce 
perfectly acceptable client-side browser-executed
web programs. In practice however, it is difficult to write Javascript which produces 
such perfect results. Javascript suffers from a number of problems. Javascript is an
interpreted language with weak/dynamic typing. Its syntax is inconsistent and verbose.
It provides no support for lazy evaluation.


Javascript supports weak, dynamic typing. Types are inferred when the program is
executed. The term 'duck typing' is used to describe such typing, where the types
of objects are inferred based on their attributes. This means that programmers need
not concern themselves with annotating the types of their functions and variables in
Javascript, which some see as a plus. This advantage however is greatly offset by
the ease with which it can be misused. Misunderstanding of javascript's typing 
frequently results in unexpected behaviour and bugs which can only be diagnosed at
runtime. 

The syntax of javascript is verbose and arguably unpleasant, the product of a different
age of language design. As such, learning to use javascript can be quite painful, tedious
and counter-intuitive for the first-time programmer. This problem persists even for
more experienced users, in the difficulty inherent in trying to find bugs or errors in
failing javascript or attempting to become familiar with an existing javascript codebase.
A number of other languages have been created in an attempt to solve this problem.

Javascript is strictly evaluated with no provisions nor means for writing lazily evaluated
programs. Lazy evaluation is an evaluation strategy consisting of call-by-need evaluation
as well as \emph{sharing}. Call-by-need evaluation allows for the evaluation of
computations to be delayed until such time as their values are needed (if ever). Sharing
is a technique where expressions are overwritten with their evaluated value after their
first evaluation meaning that the expression need not be evaluated in future when its
value is needed, improving efficiency. These concepts are common in functional languages
such as Haskell, Miranda etc. and allow for improved efficiency compared to equivalent
strictly evaluated expressions in certain circumstances. They also allow us to make use
of interesting concepts like infinite lists, better explained later.

These are problems which have long been identified in Javascript. Languages such
as Haskell, have shown that they can be solved albeit in a different programming domain.
Other languages such as CoffeeScript and Fay have attempted to fix some of these 
problems in the domain of web programming. We shall examine such existing solutions
to this problems with the hope of using them as inspiration for functional solutions
to some of the problems of browser side programming.