
\section*{Goal}
The goal of this project is to provide alternative solutions to a number of problems
identified in browser-side web programming. We will look to existing concepts in 
functional programming for inspiration when designing these solutions. The ubiquitous 
standard programming language for writing browser side web programs is javascript. 
Javascript can produce 
perfectly acceptable client-side browser-executed web programs. In practice however, 
it is difficult to write Javascript which produces such perfect results. Javascript 
suffers from a number of problems. Javascript is an
interpreted language with weak/dynamic typing. Its syntax is inconsistent and verbose.
It provides no support for lazy evaluation. As we are unable to change the semantics
of JavaScript, we shall instead use it as a target language. Our project will 
provide a means of executing a minimal strongly-typed functional language in a web
browser. Our implemenation will consist of a compiler to 
compile this functional language into an executable state and a runtime capable of 
evaluating this state to a result. The runtime will executable from within a browser.
Programs written in this language will be type-safe and our runtime will be capable
of evaluating them lazily. Our runtime will be written in JavaScript so as to be 
compatible with all major browsers and allow for the lazy evaluation of programs
written in our input language.

Javascript supports weak, dynamic typing. Types are inferred when the program is
executed. The term 'duck typing' is used to describe such typing, where the types
of objects are inferred based on their attributes. This means that programmers need
not concern themselves with annotating the types of their functions and variables in
Javascript, which some see as a plus. This advantage however is greatly offset by
the ease with which it can be misused. Misunderstanding of javascript's typing 
frequently results in unexpected behaviour and bugs which can only be diagnosed at
runtime. 

The syntax of javascript is verbose and arguably unpleasant, the product of a different
age of language design. As such, learning to use javascript can be quite painful, tedious
and counter-intuitive for the first-time programmer. This problem persists even for
more experienced users, in the difficulty inherent in trying to find bugs or errors in
failing javascript or attempting to become familiar with an existing javascript codebase.
A number of other languages have been created in an attempt to solve this problem.

Programs written in JavaScript are strictly evaluated by default. It is possible to write
lazily evaluated programs in JavaScript however doing so requires us write the underlying
lazy representations and methods ourselves. By default, JavaScript provides no means to 
write natively lazy programs. Lazy evaluation is an evaluation strategy consisting of 
call-by-need evaluation as well as \emph{sharing}. Call-by-need evaluation allows for the evaluation of
computations to be delayed until such time as their values are needed (if ever). Sharing
is a technique where expressions are overwritten with their evaluated value after their
first evaluation meaning that the expression need not be evaluated in future when its
value is needed, improving efficiency. These concepts are common in functional languages
such as Haskell, Miranda etc. and allow for improved efficiency compared to equivalent
strictly evaluated expressions in certain circumstances. They also allow us to make use
of interesting concepts like infinite lists, better explained later.

These are problems which have long been identified in Javascript~\cite{JSGoodParts}. Languages such
as Haskell, have shown that they can be solved albeit in a different programming domain.
Other languages such as CoffeeScript~\cite{Coffee} and Fay~\cite{Fay} have attempted to fix some of these 
problems in the domain of web programming. We shall examine such existing solutions
to this problems with the hope of using them as inspiration for functional solutions
to some of the problems of browser side programming.
