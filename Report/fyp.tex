\documentclass[11pt]{article}
%Gummi|062|=)

\usepackage{fancyhdr}
\usepackage{hyperref}
\title{\textbf{OpenGL Project Report}}
\author{Kieran Manning \\ 09676121}
\date{December 2012}

\newcommand{\lang}{\$LANG}

\begin{document}

\maketitle



%	i, ii etc. page numbering for contents, intro etc. etc.

\pagenumbering{roman}

\tableofcontents

\setcounter{page}{1}

\section{Acknowledgements}
Lorem ipsum

\section{Abstract: a short summary} 
Lorem ipsum

\section{Goal}
Javascript through Haskell via some form of core-like intermediate language.
Preserve laziness, typesafety, improve on syntax

\newpage

\pagenumbering{arabic}

\pagebreak

\section{Problem Space}
\subsection{Javascript Problems}
\begin{itemize}
\item Late binding
\item Weak Typing
\item Ganky syntax
\end{itemize}

\subsubsection{Syntax}
JS syntax is hella pants. Able to fix this though!
\begin{itemize}
\item Generally awkward, verbose
\item JS is Old. New ideas on syntax have emerged
\item Succinct simplicity of syntax in languages such as Python as contrast
\item Overview of ease of writing DSLs in haskell as a solution
\end{itemize}

\subsubsection{Weak Typing}
General type safety issues

\subsubsection{Late Binding}
Efficiency, runtime vs. compile time typing

\pagebreak

\section{Existing Solutions}
Fay, iTasks, GHCJS. Why these aren't what I wanted.

\subsection{Fay}
The Fay language, which lives \href{https://github.com/faylang/fay/wiki}{here}, 

\subsection{iTasks? Relevant?}

\subsection{GHCJS}

\subsection{CoffeeScript}
Worth mentioning from point of view of non-functional take on the js problem

\section{Propose solution involving Haskell}
\begin{itemize}
\item Discuss type safety, laziness, ease of haskell/JS interoperability.
\item Compare our core-level output ideas with existing solutions. Include quick chats
	  core/STG in general. Link to SPJ paper(s). Actually, maybe keep chats for later.
\item Worth comparing to ClojureScript 
\end{itemize}

\pagebreak

\section{Description of implementation and choices}

\subsection{A subset language as a Haskell DSL}
This will 

\subsection{Syntax}
One of the earlier considerations for this project, by virtue of the natural order
of implementing features in \lang, was the syntax of the developer facing input 
language (the "language" itself really). I knew from the outset that the syntax
used in javascript was not something I wished to reinvent, seeing it more as a
problem than an inspiration. 

The syntax in javascript feels like a relic of a different time. It is a common
problem brought up by programmers new to the language, and an accepted
hindrance for anyone more experience with javascript. 

Keywords such as 'new', 'var', and 'function' serve as examples of ideas on syntax
design which have become far less common since javascript was created. 'var', for
instance, adds verbosity to the language while providing little benefit. Seemingly
inspired by Scheme's 'define' concept, it adds additional cruft to declaration 
statements which are otherwise self-explanatory in languages with more minimal 
syntaxes such as Python. A declaration can be inferred sufficiently from a simple
"x = y" statement and its context. 'function' as a keyword also seems to have been
inspired by Scheme, in this case its liberal use of the 'lambda' keyword to represent
anonymous functions. Languages such as Haskell dont overcomplicate their function
defintions, content with statements as simple and declarative as 'f x y = x + y',
as well as internally scoped (fix that terminology when less tired) functions using
the where keyword etc.
Type signatures are optional but do serve to make the code in question more readable and
solve some programmer headaches (from a syntactical point of view, they obviously have other 
significant semantic effects). 

Both Fay and CoffeeScript went the minimal, Pythonic route with syntax design. Statements
such as...
\begin{itemize}
\item Variable declarations : "\(number = 42\)"
\item Function declarations : "\(square = (x) \rightarrow x * x\)"
\item List comprehensions : "\(cubes = (math.cube num \emph{for} num \emph{in} list)\)"
\end{itemize}

...in coffeescript contrast sharply with the parentheses, "function" and semicolon 
ridden expanse of comparable javascript. The Fay approach is similar in its brevity.

My approach to syntax design would be influenced by these ideas. They are reflected 
in the simple declarative statements included in my language, and were I to continue
this project beyond its current scope the same concepts would continue to be applied.


\subsection{Core/STG}
link to SPJ etc. papers. Talk about SAPL and the iTasks project.

\subsection{Representing laziness?}
Fay, thunks-as-JS-functions

\subsection{Flow control}
case statement continuations, link to paper

\subsection{Higher order functions}

















\end{document}
