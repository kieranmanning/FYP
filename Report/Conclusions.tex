\section{Conclusions}
I am mostly satisfied with the outcome of this project. The
set of features my compiler and runtime are capable of compiling
and evaluating are sufficient to express most basic concepts
one would expect in a functional language. With data constructors
we are able to express lists and user-defined types. Our Case 
statements and conditionals can be used to define more complex
flow control such as loops. Characters and strings are currently
absent but as discussed in the evaluation and further works sections
their addition would be complicated. I'm somewhat disappointed at
the lack of garbage collection as this was a feature I'd hoped to
have a chance to experiment with. As mentioned, the lack of a core
parser is unfortunate but does not detract from the more interesting
aspects of the project. 

When I started this project, I was somewhat proficient at Haskell
but could have benefitted from more experience writing
it. This project provided that experience and improved my
confidence with the language. My JavaScript abilities have
improved greatly as a result of this project, having been quite
weak at the beginning. I originally had no experience with functional
compilation and as a consequence of the research conducted through-out
it is now a topic in which I am quite interested. I look forward
to further reading on the subject and hope that I will get another
chance to apply some of what I've learned about functional compilation
and functional languages in general. 

Were I to start this project again, I probably would have implemented
a basic compiler as early as possible. I somewhat underestimated the
scope of writing a compiler for even a minimal functional language
and in the end this cost me. Writing the JavaScript runtime was 
particularly arduous. Having to switch between writing JavaScript
and Haskell really made obvious the benefits of writing in a 
langauge with a strong type system. If I was to write the runtime
again, I would probably investigate ways of modelling strong
typing in JavaScript through its use of prototypes etc. I would
also have made use of some kind of testing framework. In reality,
this would probably have meant rolling my own as JavaScript testing
frameworks have proven themselves unhelpful to me in the past. While
this would be time consuming it would have reduced many of the
headaches I experienced as a consequence of trying to track down
JavaScript bugs. 