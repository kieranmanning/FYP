\section{Problem Space}

\subsection{Typing}
JavaScript is weakly, dynamically typed. As it is an interpreted language,
there is no notion of static, compile-time typing. The types of variables
and objects are inferred at runtime based on their values, or attributes and
methods in the case of objects. This typing style is referred to as\emph{
duck typing} and similar is used in other languages such as Python. The
consequences of this typing lead to some of JavaScripts notable characteristics.
Firstly, programmers do not need to concern themselves with annotating the
types of their objects, functions etc. when declaring them and the syntax
relfects this. The statement...
\begin{center}
var x = 2;		
\end{center}
\noindent ...will, when evaluated at runtime, create a variable named x, infer
it to be of type int based on the value assigned to it and assign it the
value 2. Objects and functions are declared in a similar fashion, with their
attributes and behaviour used to determine their types. This runtime inference
exemplifies the dynamic nature of javascript typing. It also means that
type errors can only be diagnosed at runtime, generally with unhelpful 
error messages. A strongly typed language would be capable of finding such
errors at compile time. This is a somewhat imperfect argument as JavaScript
is an interpreted language, however strong typing would make it easier to
diagnose errors at runtime and allow us to perform useful type checking
ahead of time if we wished.

We can see the effects of JavaScript's weak typing in its type coercions. 
Implicit casting occurs frequently in JavaScript programs. It could be argued
that this is a useful convenience feature, though in practice such coercions
can be vague and unintuitive. One such example is arithmetic in the presence
of strings. If we call an arithmetic operator on N values,
one or more of which is a digit string, they will be cast to numerals and
the operation applied, returning a numeral result. for example

\begin{center}
	\verb!"2" * "2" => 4! \\
	\verb! 2  * "2" => 4!
\end{center}

\noindent Calling the same result on non digit strings will return a NaN and
program execution will continue (probably breaking soon). The \verb!+! operator
is even more interesting, as it is also overloaded as a string concatenation
operator. Using \verb!+! on numeral values will return a numeral value. 
Using it on some combination of numeral and string values however will break
addition associativity:

\begin{center}
	\verb!("x" + 1) + 2 => x12! \\
	\verb!"x" + (1 + 2) => x3! \\
	\verb!7 + 7 + "7" => "147"! \\
	\verb!"7" + 7 + 7 => "777"!
\end{center}

\noindent It is very common for bugs to arise in JavaScript programs where
variables have been implicitly cast to unexpected types. The
resulting program will probably break with a completely unrelated error
when some function or operator chokes on an unexpected, unintended value.
Worse yet, the program may break silently and end up in production with 
an undiscovered bug.

JavaScript is also overly forgiving of type errors when they do occur.
One such example is shown above, in the addition of non digit strings
resulting in a NaN return. Another more worrying example is the Infinity
numeric value, which occurs when a value goes outside the bounds of
a floating point number. Much like our NaN example, the program will
continue to run until it chokes on this Infinity value. This permissive
behaviour along with javascript's weak typing and implicit (often unintuitive)
casting makes it unfortunately easy to write programs which exhibit
unintended behaviour with non-existant or silent errors.
 


\subsection{Syntax}
When javascript was created, its syntax was intended to resemble that 
of C and Java. At the time, these were two predominant languages and
reusing ideas from their syntax design was intended to lessen the 
learning curve for programmers coming from C and Java backgrounds.
Since 1995 many new ideas for syntax design have appeared in more
recent languages. 

JavaScript's syntax is awkward and verbose in places. Nested 
anonymous functions for example can quickly become ugly, requiring
careful curly brace placement and indentation to remain vaguely
readable. The use of curly braces, parenthesis and semicolons to
seperate and sequence statements allows for horribly ugly, executable
code. Languages such as Python and Haskell have found ways
to overcome these problems through the use of whitespace and significant
statement placement. Their forced coding styles produce cleaner, more
standardized code which is more readable with less fluctuation from
programmer from programmer. 

A more significant problem with JavaScript can be its handling of
operators. The \verb!+! operator described above is a good
example. By default, the same symbol is used for string concatenation
as for arithmetic addition. Operator overloading is a matter of opinion,
though overloading operators as common as \verb!+! by default is probably
not the wisest or most intuitive choice. 

\subsection{Lazy Evaluation}
Javascript is a strictly evaluated language. This isn't a problem per
se and and there is no "better" choice between lazy or strict evaluation.
However, the option of lazy evaluation in browser side programming would
be nice. 

Lazy evaluation has shown itself to be useful in languages such as
Haskell. Firstly, it allows us to make use of concepts such as infinite
data structures. We could for example, create an infinite list of items
forming a recurring pattern and take as many items as we wish from this
list. Such operations in JavaScript are not possible. The syntax does
not exist to allow ease of creation of such data structures and even if
it were possible, we'd hit the obvious problem of trying to represent
such structures in a strictly evaluated language. Lazy evaluation also
provides for improved efficiency under certain circumstances. Lazily
evaluated expressions are not universally faster than their strict
counterparts but in many lazy languages the option exists to perform
operations in a strict or lazy context, allowing the programmer to choose
the better evaluation strategy for a given task. It would be great if
we could bring similar flexibility to browser-side programming.



