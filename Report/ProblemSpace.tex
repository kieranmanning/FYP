
\section{Problem Space}
An important aspect of web development is the ability to write dynamic
websites which will adapt to users requirements, input etc. This can be
achieved by writing adaptive client-side or server-side web programs
which serve dynamic content. Client-side browser executed programming 
has become  increasingly important as it is more scalable and allows 
for truly dynamic content without the need for page refreshing. The 
ubiqitous standardized programming language for client-side is JavaScript.
This is the only language which can be executed on all major browsers and 
that seems unlikely to change in the forseeable future. 

JavaScript was created in 1995 for Netscape by Brendan Eich. It is an
object-orientated, imperative, interpreted language intended originally
to be executed by a web browser (although server-side implementations
also exist, notably Node.js in recent years). Despite the name, JavaScript
is unrelated to Java. They both share a similar \verb!C++!-inspired
curly-brace syntax and object orientation but that is about as deep as
the simalirities go. JavaScript is weakly, dynamically typed. Its objects
are not class based but prototyped based. It is interpreted, meaning
that the human readable source code is executed by the browser without
any pre-processing. 

Some features have been added to JavaScript over the years but in 
general the language has not greatly changed since its inception. This
is understandably problematic. As of this moment of this writing the
language is 18 years old. Few technologies can claim to have remained
in use that long in this industry and fewer still without detractors.
As web development and websites have become increasingly more complex
problems inherent in JavaScript have started to become more obvious.
JavaScript's syntax is unappealing and verbose. Its type system is 
weak in more than name alone, leading to frequent problems with ill-
written JavaScript programs breaking silently. Many such problems 
would have been detected or prevented in languages with stronger
type systems. JavaScript lacks any concept of a module system making
inclusion difficult and naive. Programs are "included" in web pages
by linking to their sources which are basically concatenated by the
browser executing the page. Such issues existed from the first 
release of the language. As the websites and the JavaScript
programs on which they rely have grown larger, unimaginably so since
1995, these issues have become increasingly problematic.

Many people have attempted to
write alternative languages which could replace JavaScript but 
introducing new standards in the hopes of replacing old standards
usually ends as predicted. As JavaScript began as the \emph{the
standard}, so it seems it shall continue. There is some light to the
situation, however. Much time and effort has been put into making 
JavaScript run fast and it is capable of great efficiency. While 
JavaScript may be our only client side programming language, at least
it \emph{does} provide us with a standard upon which all major web 
browser vendors will agree. There have been attempts
to improve JavaScript by augmenting its feature set through the use
of user created libraries. One notable example is JQuery, a JavaScript
library which adds new features and alternative syntax to the language.
These are still only addons though, relying on the underlying semantics
and executing of the original language. 


\subsection{Typing}
JavaScript is weakly, dynamically typed. As it is an interpreted language,
there is no notion of static, compile-time typing. The types of variables
and objects are inferred at runtime based on their values, or attributes and
methods in the case of objects. This typing style is referred to as\emph{
duck typing} and similar is used in other languages such as Python. The
consequences of this typing lead to some of JavaScripts notable characteristics.
Firstly, programmers do not need to concern themselves with annotating the
types of their objects, functions etc. when declaring them and the syntax
relfects this. The statement...
\begin{center}
var x = 2;		
\end{center}
\noindent ...will, when evaluated at runtime, create a variable named x, infer
it to be of type int based on the value assigned to it and assign it the
value 2. Objects and functions are declared in a similar fashion, with their
attributes and behaviour used to determine their types. This runtime inference
exemplifies the dynamic nature of javascript typing. It also means that
type errors can only be diagnosed at runtime, generally with unhelpful 
error messages. A strongly typed language would be capable of finding such
errors at compile time. This is a somewhat imperfect argument as JavaScript
is an interpreted language, however strong typing would make it easier to
diagnose errors at runtime and allow us to perform useful type checking
ahead of time if we wished.

We can see the effects of JavaScript's weak typing in its type coercions. 
Implicit casting occurs frequently in JavaScript programs. It could be argued
that this is a useful convenience feature, though in practice such coercions
can be vague and unintuitive. One such example is arithmetic in the presence
of strings. If we call an arithmetic operator on N values,
one or more of which is a digit string, they will be cast to numerals and
the operation applied, returning a numeral result. for example

\begin{center}
	\verb!"2" * "2" => 4! \\
	\verb! 2  * "2" => 4!
\end{center}

\noindent Calling the same result on non digit strings will return a NaN and
program execution will continue (probably breaking soon). The \verb!+! operator
is even more interesting, as it is also overloaded as a string concatenation
operator. Using \verb!+! on numeral values will return a numeral value. 
Using it on some combination of numeral and string values however will break
addition associativity:

\begin{center}
	\verb!("x" + 1) + 2 => x12! \\
	\verb!"x" + (1 + 2) => x3! \\
	\verb!7 + 7 + "7" => "147"! \\
	\verb!"7" + 7 + 7 => "777"!
\end{center}

\noindent It is very common for bugs to arise in JavaScript programs where
variables have been implicitly cast to unexpected types. The
resulting program will probably break with a completely unrelated error
when some function or operator chokes on an unexpected, unintended value.
Worse yet, the program may break silently and end up in production with 
an undiscovered bug.

JavaScript is also overly forgiving of type errors when they do occur.
One such example is shown above, in the addition of non digit strings
resulting in a NaN return. Another more worrying example is the Infinity
numeric value, which occurs when a value goes outside the bounds of
a floating point number. Much like our NaN example, the program will
continue to run until it chokes on this Infinity value. This permissive
behaviour along with javascript's weak typing and implicit (often unintuitive)
casting makes it unfortunately easy to write programs which exhibit
unintended behaviour with non-existant or silent errors.
 


\subsection{Syntax}
When javascript was created, its syntax was intended to resemble that 
of C and Java. At the time, these were two predominant languages and
reusing ideas from their syntax design was intended to lessen the 
learning curve for programmers coming from C and Java backgrounds.
Since 1995 many new ideas for syntax design have appeared in more
recent languages. 

JavaScript's syntax is awkward and verbose in places. Nested 
anonymous functions for example can quickly become ugly, requiring
careful curly brace placement and indentation to remain vaguely
readable. The use of curly braces, parenthesis and semicolons to
seperate and sequence statements allows for horribly ugly, executable
code. Languages such as Python and Haskell have found ways
to overcome these problems through the use of whitespace and significant
statement placement. Their forced coding styles produce cleaner, more
standardized code which is more readable with less fluctuation from
programmer from programmer. 

A more significant problem with JavaScript can be its handling of
operators. The \verb!+! operator described above is a good
example. By default, the same symbol is used for string concatenation
as for arithmetic addition. Operator overloading is a matter of opinion,
though overloading operators as common as \verb!+! by default is probably
not the wisest or most intuitive choice. 

\subsection{Lazy Evaluation}
Javascript is a strictly evaluated language. This isn't a problem per
se and and there is no "better" choice between lazy or strict evaluation.
However, the option of lazy evaluation in browser side programming would
be nice. 

Lazy evaluation has shown itself to be useful in languages such as
Haskell. Firstly, it allows us to make use of concepts such as infinite
data structures. We could for example, create an infinite list of items
forming a recurring pattern and take as many items as we wish from this
list. Such operations in JavaScript are not possible. The syntax does
not exist to allow ease of creation of such data structures and even if
it were possible, we'd hit the obvious problem of trying to represent
such structures in a strictly evaluated language. Lazy evaluation also
provides for improved efficiency under certain circumstances. Lazily
evaluated expressions are not universally faster than their strict
counterparts but in many lazy languages the option exists to perform
operations in a strict or lazy context, allowing the programmer to choose
the better evaluation strategy for a given task. It would be great if
we could bring similar flexibility to browser-side programming.



