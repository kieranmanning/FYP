\documentclass{beamer}

\usetheme{Copenhagen}
\useinnertheme{circles}
\usecolortheme{beaver}

\title{Browser-side Functional Programming}
\subtitle{Second reader demo}
\date{April 4th 2013}
\author{Kieran Manning \\ 09676121}

\begin{document}

\section{Title}

\frame{\titlepage}

\section{Introduction}
\begin{frame}
	\frametitle{An Overview}
	The Problem? \\
	- Javascript is unpleasant but ubiquitous. Javascript 
	Suffers from..
	\begin{itemize}
		\item A lack of sane typing
		\item Verbose and inconsistent syntax
		\item A lack of laziness
	\end{itemize}
	These have been solved in languages such as Haskell, however
	it looks like we're stuck with Javascript for the forseeable 
	future. What we need is a way of bringing some functional
	inspiration to browser-side programming via Javascript.
\end{frame}

\begin{frame}
	\frametitle{Implementation}
	An approach is needed that would...
	\begin{itemize}
		\item Bring lazy evaluation and type safety to Javascript
		\item Be modular, packageable.
		\item Be as efficient as possible.
		\item Allow easy seperation of runtime evaluation from 
		  	  compilation
		\item Require no changes to Javascript itself
	\end{itemize}
\end{frame}

\begin{frame}
	\frametitle{Existing Approaches}
	\begin{itemize}
		\item Fay
		\begin{itemize}
			\item Subset of Haskell
			\item Lazy, pure, static typing
			\item Compiles to JavaScript
			\item Runtime operates at thunk level
		\end{itemize}
		\item iTasks
		\begin{itemize}
			\item Toolkit for workflow support applications
			\item Written in Clean
			\item Compiles SAPL into JS workflow apps
			
		\end{itemize}
		\item ClojureScript - Compiles clojure into readable JS
		\item CoffeeScript - JavaScript wrapper language
	\end{itemize}
\end{frame}

\begin{frame}
	\frametitle{A Plan Forms!}
	\begin{itemize}
		\item Accept Core-level language a la iTasks, take advtange of GHC.
		\item As with both iTasks and Fay, we will be working in a language
			  which already provides type safety and laziness.
		\item Aim to provide features from which we can build. Primitive
			  values, function definitions, conditionals, data constructors.
			  Ignore I/O etc. for purity and simplicity.
		\item G-Machine compiler implementation
			  to compile our parsed Core representation into an instruction
			  level language.
		\item Lastly, as with Fay, we will build a runtime in our target
			  language to evaluate these instructions while preserving the
			  lazy, type-safe semantics of our input.
	\end{itemize}
\end{frame}

\begin{frame}
	\frametitle{Core}
	Name given to a number of intermediate functional language
	representations descended from System F lambda calculus,
	consisting notably of...
	\begin{itemize}
		\item Super combinator defintions and applications
		\item Variable names
		\item Integers
		\item Data constructors
		\item Case statements
	\end{itemize}
	GHC is capable of outputting a core-like language (see
	GHC external core) which is lazy and type-checked by GHC
	itself.
\end{frame}

\begin{frame}[fragile]
	\frametitle{What does GHC ext-core look like?...}
	fac 0 = 1; fac n = ( n * fac (n - 1 ) ) becomes...
\begin{verbatim}
{fac :: Int -> Int =
  \ (d0::Int) -> %case Int d0 %of (wildX4::Int)
  {Types.Izh (d1::Int) -> %case Types.Int d1 %of (d6::Int)
  {%_ ->
  * @ Int NumInt wildX4 (
    - @ Types.Int Int (fac wildX4) 
      (Types.Izh (1::Int)
    );
  (0::Int) -> Types.Izh (1::Int)}}};
  main :: Types.Int = fac (Types.Izh (4::Int));
\end{verbatim}

\end{frame}

\begin{frame}
	\frametitle{The G-Machine}

	The G-Machine, a functional compilation strategy.
	\begin{itemize}
		\item Historically significant but somewhat outdated.
		\item Takes a core-like language and produces an initial
			  graph state.
			  and sequence of instructions...
		\item ...to be evaluated by a minimal runtime.
		\item Designed to allow for easy adaptation of runtime
			  in target languages.
		\item More efficient and modular than alternative Template
			  Instantation approach, also explored.
	\end{itemize}
	
\end{frame}
	
\begin{frame}
	\frametitle{A Graph Evaluation Runtime in Javascript}
	A runtime was required which would fulfill the following
	requirements:
	\begin{itemize}
		\item Take a representation of a G-Machine compiled
			  graph state.
		\item Take a list of G-Machine evaluation instructions.
		\item Understand how to evaluate these instructions.
		\item Apply these instructions to the supplied graph and
		\item Return the resultant state of the graph.
	\end{itemize}	
	
\end{frame}

\section{Implementation}

\begin{frame}
	\frametitle{Implementation, an overview}
	\begin{enumerate}
	\item We take GHC's external core, which we parse into a
	Haskell ADT representing the language. 
	\item We then use the G-Machine compilation schemes to 
	compile this ADT into 
		\begin{itemize}
			\item A heap representing a graph and a stack
			to index this heap
			\item A sequence of instructions to be executed 
			by our runtime.
		\end{itemize}
	\item We package this state and instructions into a 
	form javascript can handle and then pass to the runtime.
	\item The runtime, written in javascript, executes until
	we have reached a final state (or something breaks...) 
	leaving us with a final value on top of the stack.
	\end{enumerate}
\end{frame}

\begin{frame}
	\frametitle{G-Machine implemenation}
	\begin{itemize}
	\item Implementing a G-Machine based graph compiler is 
	a solved
	problem, and rather than re-inventing the wheel I decided
	to work off the implementation given in Implementing 
	Functional Languages: A Tutorial by Simon Peyton Jones
	and David Lester. 
	\item Once our core representation has been compiled to
	a graph, it still needs to be serialized into a form that
	Javascrpt can handle. Haskell2JS.hs takes a state consisting
	of a stack, heap, list of globals and sequence of
	instructions and returns the equivalent represented
	using Javascript arrays and objects.
	\end{itemize}
	
\end{frame}

\begin{frame}[fragile]
	\frametitle{Runtime Implementation}
	Initial considerations
	\begin{itemize}
		\item Data representations? \\
		- Function objects to represent
		instructions, nodes: 
		\begin{verbatim}
		var x = new function PushInt(Int){this.n = Int;}
		\end{verbatim}
		- Arrays, objects to represent our state 
		\begin{verbatim}
		var GmHeap = {objCount, freeAddrs, addrObjMap};
		\end{verbatim}
		\item Instruction evaluation? \\
		- JS functions operating on a passed state 
		\begin{verbatim}
		function pushint(n, state){push, return state';}	
		\end{verbatim}
	\end{itemize}
\end{frame}

\begin{frame}[fragile]
	\frametitle{Runtime Evaluation: Initial State}
	Initially, we have
	\begin{itemize}
		\item A heap, containing compiled supercombinators
			  representing our main function, additional 
			  functions and language primitives ( +, -, if ...)
		\item An empty stack
		\item A code sequence consisting of
		\begin{verbatim} 
		[ PushGlobal "main", Eval ]
		\end{verbatim} 
		\item And a list of globals, mapping function names
			  to heap addresses
		\begin{verbatim}
		[("+", 5), ("main", 1)...]
		\end{verbatim}
		\item A dump to handle arithmetic operations when not in
		WHNF
	\end{itemize}
	Shall show such an example at some point...
\end{frame}

\begin{frame}[fragile]
	\frametitle{Runtime Evaluation: Instructions Overview}
	The expected instructions for a G-Machine runtime.
	Slide, Push etc. to manipulate stack. Mkap, Eval, Unwind
	etc. to evaluate expressions. All executing on a passed
	state and returning a new state...
\begin{verbatim}
function push(N, xState){
    var State 		= xState;	// <- JS state issues
    var stack 		= getStack(State);
    var newStack	= [stack[N]].concat(stack);
    var newState 	= putStack(newStack, State);
    return newState; 
}
\end{verbatim}		
\end{frame}

\begin{frame}[fragile]
	\frametitle{Runtime Evaluation: Nodes Overview}
	Our runtime must accept and evaluate the usual G-Machine
	graph nodes: indirections, applications, ints, globals,
	constructors. We represent these in the abstract as functions
	which we can instantiate as function objects representing
	graphs nodes, values.

\begin{verbatim}
function NConstr(t, a){
    this.t = t;
    this.a = a;
}

[newHeap, addr] = hAlloc(heap, new NConstr(t, addrs));
\end{verbatim}

\end{frame}

\begin{frame}[fragile]
	\frametitle{Runtime Evaluation: Evaluation Strategy}
	Our Evaluation Strategy
	\begin{itemize}
		\item Encounter Pushglobal "main", push addr to stack
		\item Eval and Unwind stack, place main instructions in 
		code sequence
		\item Push arguments of first function in main to stack,
		followed by behaviour of first function and sufficient 
		Mkap instructions as to form an application tree.
		\item Defer state to dump if necessary.
		\item Reduce this expression and update root node with 
		value
		\item Rinse and repeat until code sequence empty. 
	\end{itemize}
\end{frame}

\begin{frame}
	\frametitle{Features}
	\begin{itemize}
		\item Laziness \\
		Our runtime supports laziness, including redex updates
		and call-by-need evaluation. 
		\item Function defintions and calls
		\item Arithmetic 
		\item Conditionals
		\item Data Constructors (...kinda)
	\end{itemize}
		
\end{frame}

\section{Conclusions}

\begin{frame}
	\frametitle{Conclusions}
	Some problems
	\begin{itemize}
		\item Core versioning issues
		\item Time management
		\item Javascript in general (if only someone would
		      write...)
	\end{itemize}	
	Had I more time...
	\begin{itemize}
		\item Cleaner heap implementation
		\item More efficient graph compiler implementation
		\item More efficient Javascript
		\item Core parser would be nice but unexciting
		\item Continue runtime to completion (although it's not
			  far off)
	\end{itemize}
	
\end{frame}

\end{document}

















